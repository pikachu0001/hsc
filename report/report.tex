\documentclass[12pt]{article}
\usepackage[english]{babel}
\usepackage[utf8x]{inputenc}
\usepackage{graphicx}
\usepackage{color}
\usepackage{xcolor}
\usepackage{listings}
\usepackage[hyphens]{url}
\usepackage{hyperref}
\hypersetup{
    colorlinks=true,     
    urlcolor=blue,
}
\lstset{language=Pascal}
\lstset{showstringspaces=false}
% http://users.ecs.soton.ac.uk/srg/softwaretools/document/start/listings.pdf
\begin{document}
\begin{titlepage}
\center




%----------------------------------------------------------------------------------------
%	HEADING SECTIONS
%----------------------------------------------------------------------------------------
\textsc{\LARGE Free University of Bozen - Bolzano}\\[1.5cm]
\textsc{\Large Formal Languages and Compiler}\\[0.5cm]
\textsc{\large Compiler Project}\\[0.5cm]




%----------------------------------------------------------------------------------------
%	TITLE SECTION
%----------------------------------------------------------------------------------------
\vspace{2.5cm}
\hrule
\vspace{1cm}
{\huge \bfseries HSC}
\vspace{1cm}
\hrule
\vspace{3.5cm}




%----------------------------------------------------------------------------------------
%	AUTHOR SECTION
%----------------------------------------------------------------------------------------
\begin{minipage}{0.95\textwidth}
\begin{flushleft} \large
\emph{Author:}
\newline
\newline
Giorgio \textsc{Musciagna} \hfill gmusciagna@unibz.it
\newline
Ravinder \textsc{Singh} \hfill rsingh@unibz.it
\end{flushleft}
\end{minipage}




%----------------------------------------------------------------------------------------
%	DATE SECTION
%----------------------------------------------------------------------------------------
\vfill
{\large \today}




%----------------------------------------------------------------------------------------
\end{titlepage}




%----------------------------------------------------------------------------------------
%	INTRODUCTION
%----------------------------------------------------------------------------------------
\section{Introduction}
\vspace{0.5cm}
The aim of the project is the realization of a compiler for a particular language by means of lex and yacc. A compiler is a program which converts a program written in a language, usually denoted as source, into an equivalent program written in another language, usually denoted as target language. The process of compilation comprises two parts: analysis and syntesis.\\\\
\textit{Analysis}: the source program is read and its structure is analysed.\\
\textit{Syntesis}: an intermediate code is generated from the intermediate representation of the language returned at the end of the analysis part. An optimization process will occur and the target program will be eventually generated.\\\\
The project is focused on the first part of the compilation only i.e. the analysis part which consists of two subtasks:
\begin{enumerate}
\item Split the source program and categorize each sequence of characters into tokens by reading it left-to-right. Regular expression are used in order to tokenize the source program. (Lex)
\item Find the hierarchical structure of the source program expressed by recursive rules reflecting a context-free grammar. Moreover, additional information to give meaning to the program are computed involving adding information to the \textit{symbol table} and type checking. (Yacc)
\end{enumerate}
\vspace{1.7cm}




%----------------------------------------------------------------------------------------
%	SYMBOL TABLE
%----------------------------------------------------------------------------------------
\section{Symbol table}
The symbol table is a data structure whose aim is to save additional information we might want to preserve in order to give meaning to the program. It is usually shared among lex and yacc. During the lexical analysis the code is tokenized and categorized. On the other hand, this process of categorization usually results in a loss of information related to the categorized token such as name, value. As a result, a symbol table could be used in order to preseve this information we might lose. Additional attributes will be later on added during the parsing phase such as type, scope, etc. which could be only expressed by the hierchical structure of the program. In this project, the symbol table is a linked list where each node, usually a variable, is denoted by the following four attributes: name, type, value and scope level. Here an example follows.
\vspace{0.3cm}
\begin{table}[h!]
\centering
\begin{tabular}{c|c|c|c}
price & real & 140.05 & 1 \\
done & boolean & 1 & 0 \\
price & real & 9.2 & 0 \\
rate & real & 12.57 & 0 \\
\hline
Name & Type & Value & Scope
\end{tabular}
\end{table}
\vspace{1.7cm}





%----------------------------------------------------------------------------------------
%	FEATURES OF DEVELOPED COMPILER
%----------------------------------------------------------------------------------------
\section{Features}
The main features of the developed language are the following:
\begin{itemize}
\item Variable declaration
\item Variable assignment
\item Printing (strings and expressions)
\item Typing (real and boolean)
\item Operations (arithmetic and boolean)
\item Scoping
\item Comments
\item Strings and meta chars
\end{itemize}



\subsection{General information}
Each program as usual consists of a list of statements. Moreover, reserved words and special chars are used in order to let the user express some particular commands. Here some general rules which globally apply to the language:
\begin{enumerate}
\item Statement delimiter: each statement is delimited by \texttt{\textbf{newline}}.
\item Stop execution: with the command \texttt{\textbf{halt}} the execution of the program will be interrupted.
\item Strings: literals containing meta chars are also captured, delimited by single quotes. To use them as part of the string type two consecutive single quotes \texttt{''}. Newlines and tabs can be used via \texttt{\textbackslash n} and \texttt{\textbackslash t}.
\end{enumerate}
Interesting technical info: regarding multiple declaration (see * below) a right recursive rule has been used in the yacc source file. The main reason is because the type token is at the end of the statement, therefore we have to accumulate all the variable names on the stack. Then, once the end has been reached, reductions will occur and type will be set for each variable name.
\vspace{2cm}



\subsection{Syntax and examples}
Here some examples showing the syntax of the developed language\\\\\\
\textbf{Variable declaration}: single and multiple declaration. *
\begin{lstlisting}[frame=single]
var pippo : real
var pluto : boolean
var x, y, z, w : boolean
\end{lstlisting}
\pagebreak
\textbf{Variable assignment}: after or during declaration.
\begin{lstlisting}[frame=single]
var pluto : boolean
pluto := 40 > 20
var paperino : real = 0.25
\end{lstlisting}
\vspace{1.8cm}
\textbf{Comments}.
\begin{lstlisting}[frame=single]
x := 10 (* This is a
multiline comment *)
\end{lstlisting}
\vspace{1.8cm}
\textbf{Printing}. Two versions available: \texttt{write} and \texttt{writeln}. The latter puts a newline afterwards. It is possible to print boolean or arithmetic expressions and strings. Moreover you can print a list of expressions or strings using just one single command. Writeln could also just be used without args to just insert a newline.
\vspace{0.1cm}
\begin{lstlisting}[frame=single]
writeln	'hello ''world'', how are you?\nFine... "Awwww"'
write -2 * -4
writeln
writeln
writeln 'Euler''s num is ', math.e, '\tPI is ', math.pi

(*
	hello 'world', how are you?
	Fine... "Awww"
	8.000000
	
	Euler's num is 2.718281     PI is 3.1415927
*)
\end{lstlisting}
\vspace{2cm}
\textbf{Operations}. Arithmetic and boolean operations. Some constants are also available: Pi and the Euler's number.
\begin{lstlisting}[frame=single]
pippo := 2
pippo := -pippo + math.e * 40 / math.pi
writeln 'Is pippo less or equals to 2? ', pippo <= 2 
pluto := true
pluto := !pluto && (!true || false) && true
\end{lstlisting}
\vspace{2cm}
\textbf{Scoping}. Scoping is also part of the language. Create a scope by enclosing some statements via reserved keywords \texttt{begin} and \texttt{end}. The scope of a variable and consequently its visibility starts from its declaration until the end of the block scope it belongs to. If a block contains a second block in which a variable is redeclared, then this variable hides the previous variable till the end of the second scope. As soon as the inner block ends, then the first declaration will be available.
\begin{lstlisting}[frame=single]
x := 1
writeln x                (* here x has value 1 *)
begin
   x := 12.5
   writeln x             (* here x has value 12.5 *)
   var x : real
   x := 2
   writeln x             (* here x has value 2 *)
   begin
      var x : boolean
      x := true
      writeln x          (* here x has value true *)
   end
   writeln x             (* here x has value 2 *)
end
writeln x                (* here x has value 12.5 *)
\end{lstlisting}
\vspace{1.5cm}




%----------------------------------------------------------------------------------------
%	References
%----------------------------------------------------------------------------------------
\section{References}
Unibz course webpage:\\
\url{http://www.inf.unibz.it/~artale/Compiler/compiler.htm}\\\\
Helpful online resources:\\
\url{http://www.cs.man.ac.uk/~pjj/cs212/ex2_str_comm.html}\\
\url{http://epaperpress.com/lexandyacc/str.html}\\
\url{https://www.programiz.com/c-programming/c-strings}\\
\url{http://www.learn-c.org/en/Linked_lists}\\
\url{https://stackoverflow.com/questions/1653958/why-are-ifndef-and-define-used-in-c-header-files}\\
\url{https://www.youtube.com/watch?v=EzBTm73_oU8}\\
\url{https://stackoverflow.com/questions/7751366/malloc-memory-for-c-string-inside-a-structure}
\url{https://stackoverflow.com/questions/3219393/stdlib-and-colored-output-in-c}




%----------------------------------------------------------------------------------------
\end{document}



